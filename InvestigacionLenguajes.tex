% !TEX TS-program = pdflatex
% !TEX encoding = UTF-8 Unicode

% This is a simple template for a LaTeX document using the "article" class.
% See "book", "report", "letter" for other types of document.

\documentclass[11pt]{article} % use larger type; default would be 10pt

\usepackage[utf8]{inputenc} % set input encoding (not needed with XeLaTeX)

%%% Examples of Article customizations
% These packages are optional, depending whether you want the features they provide.
% See the LaTeX Companion or other references for full information.

%%% PAGE DIMENSIONS
\usepackage{geometry} % to change the page dimensions
\geometry{a4paper} % or letterpaper (US) or a5paper or....
% \geometry{margin=2in} % for example, change the margins to 2 inches all round
% \geometry{landscape} % set up the page for landscape
%   read geometry.pdf for detailed page layout information

\usepackage{graphicx} % support the \includegraphics command and options

% \usepackage[parfill]{parskip} % Activate to begin paragraphs with an empty line rather than an indent

%%% PACKAGES
\usepackage{booktabs} % for much better looking tables
\usepackage{array} % for better arrays (eg matrices) in maths
%\usepackage{paralist} % very flexible & customisable lists (eg. enumerate/itemize, etc.)
\usepackage{verbatim} % adds environment for commenting out blocks of text & for better verbatim
\usepackage{subfig} % make it possible to include more than one captioned figure/table in a single float
% These packages are all incorporated in the memoir class to one degree or another...

%%% HEADERS & FOOTERS
\usepackage{fancyhdr} % This should be set AFTER setting up the page geometry
\pagestyle{fancy} % options: empty , plain , fancy
\renewcommand{\headrulewidth}{0pt} % customise the layout...
\lhead{}\chead{}\rhead{}
\lfoot{}\cfoot{\thepage}\rfoot{}

%%% SECTION TITLE APPEARANCE
\usepackage{sectsty}
\allsectionsfont{\sffamily\mdseries\upshape} % (See the fntguide.pdf for font help)
% (This matches ConTeXt defaults)

%%% ToC (table of contents) APPEARANCE
\usepackage[nottoc,notlof,notlot]{tocbibind} % Put the bibliography in the ToC
\usepackage[titles,subfigure]{tocloft} % Alter the style of the Table of Contents
\renewcommand{\cftsecfont}{\rmfamily\mdseries\upshape}
\renewcommand{\cftsecpagefont}{\rmfamily\mdseries\upshape} % No bold!

%%% END Article customizations

\usepackage[spanish]{babel}
\usepackage{listings} 
%%% The "real" document content comes below...

\title{Investigación de Lenguajes - Ruby}
\author{-Cristina Barreno \\ -Jordy Vasquez \\ -Sixto Castro}
%\date{} % Activate to display a given date or no date (if empty),
         % otherwise the current date is printed 

\begin{document}
\maketitle
%\tableofcontents % No hace falta un TOC en un artículo corto

\section{Introducción}
 Hara Jordy
yo jajajja

\section{Características}

\begin{itemize}

      \item  Ruby es un lenguaje multiplataforma (altamente portable).
      \item  Es fácil de escribir. 
      \item  Es un lenguaje interpretado.
      \item  Es un lenguaje orientado a objetos. Al igual que Java, cada tipo de dato es un objeto.
      \item  Es una mezcla de varios lenguajes tales como Perl, Python, Smalltalk y otros, que definen a Ruby como un lenguaje que integra la programación funcional e imperativa.
      \item  Posee expresiones similares(nivel de lenguaje) a las de Perl
      \item  En Ruby, los métodos se pueden o no ser parte de una clase. Puede ser declarado en cualquier parte del archivo.
      \item  Es capaz de manejar excepciones.
      \item  Se puede manejar hilos (multihilos) sin depender del sistema operativo.

\end{itemize}

\section{Historia}
Hace Jordy

\section{Tutorial de Instalación}
Hace Cristina

\section{Hola Mundo y otros Programas Introductorios}
\begin{itemize}
      \item {\bf Crear un array}\\
	\hspace*{7mm} @names = Array.new\\
      \item {\bf Método [ ]} \\
	\hspace*{7mm} @names = ["ivan", "luis", "pedro"] \\
      \item {\bf Método: +}\\
	\hspace*{7mm} Agregar contenido de un Arrray al final de otro Array.\\
	\hspace*{7mm}@clientes = ["Juan", "Luis", "Roberto"]\\
	\hspace*{7mm}@clientes2 = [ "Pierre", "Joe", "Rachel" ]\\
	\hspace*{7mm}@clientes + @clientes2 \\
           \hspace*{7mm} Resultado: ["Juan", "Luis", "Roberto", "Pierre", "Joe", "Rachel"]\\
     \item {\bf Ordenar un arreglo}\\
	\hspace*{7mm} notasdeberes = [10,7,5,8]\\
	\hspace*{7mm} notasdeberes.sort\\
    \item  {\bf Obtener el tamaño de un arreglo}\\
	\hspace*{7mm} notasdeberes.size()\\

     \item  {\bf Recorrer un arreglo: each}\\
    	\hspace*{7mm} equipos = ["Barcelona","Milan","PSG","Monaco"]\\
	\hspace*{7mm} equipos.each do equipo\\
 	\hspace*{14mm} puts equipo\\
	\hspace*{7mm} end\\
	
     El método 'each' lo que va hacer es recorrer cada elemento del arreglo equipos. En cada iteración, cada elemento del array se guarda en la variable equipo y se va imprimir cada elemento(cadena) del arreglo equipos ya que se encuentra la función puts que se encarga de        imprimir cada cadena.

     \item {\bf Recorrer el arreglo: for in}\\
	\hspace*{7mm} for i in equipos\\
  	\hspace*{14mm}  puts i\\
	\hspace*{7mm} end\\
     \hspace*{7mm} Este caso es parecido al ejercicio anterior, sino que ahora se utiliza un ciclo 'for' para recorrer el arreglo. En cada iteración, la variable i toma el valor actual del array equipos, y luego va imprimir dicho elmento ya que se hace un put en cada iteración.\\




\end{itemize}







\end{document}
